\documentclass[twoside,11pt]{cergdoc}
\usepackage{graphicx}
\usepackage{amsmath}

\newcommand{\ITwoC}{I\textsuperscript{2}C }

\begin{document}

\title{XXBX Software -- XBS}
\subtitle{XBS User Guide v1.0}
\author{Matthew R. Carter \and Raghurama R. Velegala \and Jens-Peter Kaps}
\email{jkaps@gmu.edu}
\affiliation{George Mason University\\
  Fairfax, Virginia}
\topicpic{xxbx-slide}

\maketitle

\tableofcontents

% -----------------------------------------------------------------------------
\chapter{Introduction}
% -----------------------------------------------------------------------------
The XBS was rewritten in Python 3, away from a mix of shell script and
Perl that it was originally written in. The primary reason for this was because
we wanted to add the ability to resume runs if a failure occurred, and we felt
it was easier to work with a full object-oriented scripting language with
libraries in order to add this functionality. We were also more familiar with
Python then we were with Perl. 

% -----------------------------------------------------------------------------
\chapter{Build Environment}
% -----------------------------------------------------------------------------
  \section{Build Process}
The builds use GNU make to build
the codebase. Multiple makefiles are
simultaneously used to do separate parallel
builds thus speeding up the build process
and not blocking by stalled builds. Rebuilds
do not require recompiling everything from
scratch, only the code that changed or that
was interrupted.

  \section{Toolchain Integration}
    \subsection{TI MSP430 and MSP432}
    \subsection{ST ARM}
% -----------------------------------------------------------------------------
\chapter{Benchmarking}
% -----------------------------------------------------------------------------
Resuming interrupted runs was something we felt was desirable, as failure could
interrupt a test run, and we did not want to have to rerun the entire test run
from scratch. This was implemented but necessitated large rewrites in order to
save state and prevent duplicate runs.

  \section{Setup}
XXBX are mostly compatible with
SUPERCOP’s output and use its algopacks,
which are collections of different crypto
algorithms (e.g. hash, AEAD, one-time
authentication, signature algorithms, etc.)
that itself uses and are updated with its own
development[6]. Algopacks is imported into
the microcontrollers by ./import\_algopack
external/supercop command. Then the
checksum of the crypto is acquired by using
python/xbx/dirchecksum.py
algobase/crypto\_core/aes128encrypt/ in
xbs\_xbd directory. The file impl\_conf.ini is
updated with the checksum of the crypto
acquired from the algopacks. Then
config.ini is updated to select the desired
platform and algorithms.

    \subsection{Algorithm Selection}
The algorithm is selected in config.ini
file. The file config.ini specifies whitelists
and blacklists of implementations,
the target operation, the target platform,
paths, parameters, dependencies, etc.
After this compile.py is ran using
./compile.py command on Linux platform.
This creates the file data.db , used to store
results, configuration, and other detailed
information.
If a whitelist exists in config.ini ,
only the implementations in this list will be
built (and later run), otherwise all
implementations available in the specified
operation and primitives are built, excluding
those specified in the blacklist. The compile
script generates the makefiles and header
files for each implementation and builds the
code for the XBD for each implementation
x {compiler, flag}. The compiler flag to use
is defined per device, for example the mspexp430f5529lp
board uses
xbs\_xbd/platforms/mspexp430f5529lp\_16mhz/c\_compilers file to
update the compiler name and flags to test
one line per configuration. Once a build
succeeds, the execute.py script is run. The
configuration used from the last build is
loaded from data.db , if config.ini has not
changed between the last build and calling
execute.py. Behavior is undefined if
config.ini or blacklists or code are modified
between calls of compile.py and execute.py.
For each successfully compiled
implementation, the binary
is loaded into the XBD and tests are run
The test case configuration for this are
•! Message (128 bytes)
•! Associated Data (Associated
Data+Message must be less than
2048 bytes)
•! Ciphertext, which is the size of
Message+Associated Data plus
additional bytes for authentication.
Additional bytes are limited to 128
bytes.
•! Secret Bytes (64 bytes)
•! Public Bytes (64 bytes)
•! Key Bytes (64 bytes)
This test case limits were configured
only for encryption.


    \subsection{Target Selection}
    \subsection{Compiler and Linker Options}
Crosstool-NG: crosstool-NG is a set of
scripts that can build a toolchain from
scratch. GCC (GNU Compiler
Collection) compiler was used because it
supported the ARM chip on the EKTM4C1294XL.
We decided to use
crosstool-ng in order to build GCC,
along with newlib, which provides an
implementation of the standard C
library.

TivaWare: The TivaWare™ software for
C Series is an extensive suite of software
tools designed to simplify and speed
development of Tiva C Series-based
MCU applications. It was used for the
EK-TM4C1294XL board.

We support Link time Optimization (LTO), which looks at the entire
generated binary when optimizing during the linking phase as opposed to a single
file, This allows for more aggressive inlining and code-size reduction. However
the aggressive optimizations may often break functionality and make debugging
extremely difficult due to the aggressive inlining- stepping through the
generated assembly is required. 

  \section{Verification}
As with the original XBX, primitive implementations are tested for correctness.
We reused the SUPERCOP primitive implementation verification code. This code has
options for large and small tests. As we are operating in a resource constrained
environment, we only use the small tests, that handle lengths up to 128 bytes. 

For hash functions, correct results are verified by mixing in the
hash function output for various sizes into an instance of Salsa20
\cite{salsa20}, instead of chaining the outputs as with original XBD. The output
of this is later compared with a known value by the XBS.
Deterministic results are also checked for, as well as if 
the implementation can handle the output being placed in the same buffer as the
input and buffers remain within bounds. 

For AEAD functions, the ciphertext is fed into the Salsa20 function, and is
decrypted with the plaintext and associated data also fed into the
Salsa20 function for checking by the XBS. In addition, the decrypted message and
associated data is compared to the original message and associated data. 
In addition to the correctness verifications, the implementation is checked to
see if the output is deterministic, if buffers remain within bounds, overlap of
input and output buffers is handled if specified by the implementation, and if
forgeries are detected.

The XBD to XBH (and vice-versa) communication code for sending and receiving
large data over multiple \ITwoC packets was refactored to reuse the same code
over multiple commands. 

 %------- merge above (John) with below (Girmay)

The XXBX reuses the SUPERCOP
primitive implementation verification code
to test primitive implementations for
accuracy. Small tests, that handle lengths up
to 128 bytes were used to test because of
resource limitations. For hash functions,
correct results are verified by mixing in the
hash function output for various sizes into
an instance of Salsa20 [7], instead of
chaining the outputs as with original XBD.
The output of this is later compared
with a known value by the XBS.
Deterministic results are also checked for, as
well as if the implementation can handle the
output being placed in the same buffer as the
input and buffers remain within bounds. For
AEAD functions, the ciphertext is fed into
the Salsa20 function, and is decrypted with
the plaintext and associated data also fed
into the Salsa20 function for checking by the
XBS.
In addition, the decrypted message
and associated data is compared to the
original message and associated data. In
addition to the correctness verifications, the
implementation is checked to see if the
output is deterministic, if buffers remain
within bounds, overlap of input and output
buffers is handled if specified by the
implementation, and if forgeries are
detected.
This was also verified by running the
a select few of algorithms on a
microcontroller using Code Composer.
These algorithms were implemented,
compiled and
debugged in the Code Composer Studio IDE
configured to work on MSP 430FF5529.


  \section{Results}
Results can be examined by opening the generated SQLite database
(\texttt{data.db}) in a SQLite browser application, such as DB Browser for
SQLite \cite{sqlitebrowser}. For analysis, the SQLAlchemy objects can be
manipulated directly in an IPython \cite{ipython} notebook. IPython is an
interactive python environment, with an interface similar to Maple or SageMath
with ``notebooks.'' The schema is listed in appendix \ref{app:schema}.

  \section{Persisting state}
In order to persist the state of the runs and the actual results, we decided
to store this information into a SQLite \cite{sqlite} database. We used
SQLAlchemy \cite{sqlalchemy} to simplify the task of persisting the state of the
XBS. SQLAlchemy is an object-relational mapper (ORM) that maps Python objects to
SQL tables and rows, without having to directly write SQL. We used this library
as it is significantly simpler than using SQL directly.


% -----------------------------------------------------------------------------
\chapter{Results Database}
% -----------------------------------------------------------------------------
XXBX results can be analyzed by
opening the generated SQLite database
(data.db ) in a SQLite browser application,
such as DB Browser for SQLite [8]. For
analysis, the SQLAlchemy objects can be
manipulated directly in an IPython [11]
notebook. IPython is an interactive python
environment, with an interface similar to
Maple or SageMath with “notebooks.”
Information was saved into a SQLite [12]
database.
We used SQLAlchemy [13] to
simplify the task of persisting the state of the
XBS. SQLAlchemy is an object-relational
mapper (ORM) that maps Python objects to
SQL tables and rows, without having to
directly write SQL. We used this library as it
is significantly simpler than using SQL
directly.
Using the generated SQLite database
(data.db) we can see the table values busing
SQLbrowser called DB Browser. DB
Browser shows the tables, and the column
values. An example is provided below.

% -----------------------------------------------------------------------------
\begin{appendix}
\chapter{Database Table Description}
\end{appendix}


\end{document}
